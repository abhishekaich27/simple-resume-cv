% !TEX TS-program = xelatex
% !TEX encoding = UTF-8 Unicode
% -*- coding: UTF-8; -*-
% vim: set fenc=utf-8

%%%%%%%%%%%%%%%%%%%%%%%%%%%%%%%%%%%%%%%%%%%%%%%%%%%%%%%%%%%%%%%%%
%% CV.tex
%% <https://github.com/zachscrivena/simple-resume-cv>
%% This is free and unencumbered software released into the
%% public domain; see <http://unlicense.org> for details.
%%%%%%%%%%%%%%%%%%%%%%%%%%%%%%%%%%%%%%%%%%%%%%%%%%%%%%%%%%%%%%%%%

% See "README.md" for instructions on compiling this document.

\documentclass[letterpaper,MMMyyyy,nonstopmode]{simpleresumecv}
% Class options:
% a4paper, letterpaper, nonstopmode, draftmode
% MMMyyyy, ddMMMyyyy, MMMMyyyy, ddMMMMyyyy, yyyyMMdd, yyyyMM, yyyy

%%%%%%%%%%%%%%%%%%%%%%%%%%%%%%%%%%%%%%%%%%%%%%%%%%%%%%%%%%%%%%%%%
%% PREAMBLE.
%%%%%%%%%%%%%%%%%%%%%%%%%%%%%%%%%%%%%%%%%%%%%%%%%%%%%%%%%%%%%%%%%

% CV Info (to be customized).
\newcommand{\CVAuthor}{Abhishek Aich}
\newcommand{\CVTitle}{Abhishek's CV for Academics}
\newcommand{\CVNote}{CV compiled on {\today} for Academic purposes}
\newcommand{\CVWebpage}{https://abhishekaich27.github.io}
\newcommand{\CVLinkedIn}{https://www.linkedin.com/in/abhishekaich27/}
\newcommand{\CVGitHub}{https://github.com/abhishekaich27}


% PDF settings and properties.
\hypersetup{
pdftitle={\CVTitle},
pdfauthor={\CVAuthor},
pdfsubject={\CVWebpage},
pdfcreator={XeLaTeX},
pdfproducer={},
pdfkeywords={},
unicode=true,
pdfstartview=FitH,
pdfpagelayout=OneColumn,
pdfpagemode=UseOutlines,
hidelinks,
breaklinks}

% Shorthand.
\newcommand{\Code}[1]{\mbox{\textbf{#1}}}
\newcommand{\CodeCommand}[1]{\mbox{\textbf{\textbackslash{#1}}}}

%%%%%%%%%%%%%%%%%%%%%%%%%%%%%%%%%%%%%%%%%%%%%%%%%%%%%%%%%%%%%%%%%
%% ACTUAL DOCUMENT.
%%%%%%%%%%%%%%%%%%%%%%%%%%%%%%%%%%%%%%%%%%%%%%%%%%%%%%%%%%%%%%%%%

\begin{document}

%%%%%%%%%%%%%%%
% TITLE BLOCK %
%%%%%%%%%%%%%%%

\Title{\CVAuthor}

\begin{SubTitle}
\href{https://goo.gl/maps/zEn5mL4zoZM2}
{\faMapMarker ~WCH 371, University of California Riverside, CA 92521, USA}
\par
\href{mailto:aaich001@ucr.edu}
{aaich001@ucr.edu}
%\,\SubBulletSymbol\,
%+1\,(951)\,396-8731
\,\SubBulletSymbol\,
\href{\CVWebpage}
{\faHome}
\,\SubBulletSymbol\,
\href{\CVLinkedIn}
{\faLinkedinSquare}
\,\SubBulletSymbol\,
\href{\CVGitHub}
{\faGithubSquare}
\end{SubTitle}

\begin{Body}
%%%%%%%%%%%%%%%%%%%%%%%%
%% RESEARCH INTERESTS %%
%%%%%%%%%%%%%%%%%%%%%%%%

\Section
{Research Interests}
{Research Interests}
{PDF:Research Interests}

\Entry
Computer Vision, Deep Learning, and Sparse Signal Optimization
\begin{Detail}
\SubBulletItem
Specific Interests: Video Reconstruction, Continual Learning, Person Re-identification
\end{Detail}
\BigGap
%%%%%%%%%%%%%%%
%% EDUCATION %%
%%%%%%%%%%%%%%%

\Section
{Education}
{Education}
{PDF:Education}

\Entry
\href{http://www.ucr.edu}
{\textbf{University of California}},
Riverside, CA, USA

\Gap
\BulletItem
Ph.D. in
\href{https://www.ee.ucr.edu}
{Electrical and Computer Engineering}
\hfill
\DatestampYMD{2018}{09}{15}  --
Present
\begin{Detail}
%\SubBulletItem
%Thesis:
%\href{http://www.example.com/my-phd-thesis}
%{A Statistical Approach to Quantifying Climate Change}
\SubBulletItem
Adviser:
\href{https://www.engr.ucr.edu/people/amitroychowdhury.html}
{Dr.~Amit K. Roy-Chowdhury}
\SubBulletItem
GPA: 3.82 / 4.00
\end{Detail}

\BigGap
\Entry
\href{http://www.nitt.edu}
{\textbf{National Institute of Technology}},
Tiruchirappalli, Tamil Nadu, India

\Gap
\BulletItem
M.S. in
\href{https://www.nitt.edu/home/academics/departments/ece/}
{Electronics and Communication Engineering}
\hfill
\DatestampY{2016} --
\DatestampY{2018}
\begin{Detail}
\SubBulletItem
Thesis:
Exploiting Sparsity for Direction of Arrival Estimation Algorithms in Linear Array
%\href{http://www.example.com/my-phd-thesis}
%{A Statistical Approach to Quantifying Climate Change}
\SubBulletItem
Adviser:
\href{https://www.nitt.edu/home/academics/departments/ece/faculty/associate_professor/pp/}
{Dr.~P. Palanisamy}
\SubBulletItem
GPA: 8.80 / 10.00
\end{Detail}

\BigGap
\Entry
\href{http://www.bput.ac.in}
{\textbf{Biju Patnaik University of Technology}},
Rourkela, Odisha, India

\Gap
\BulletItem
B.Tech. in
\href{https://silicon.ac.in/sitbbsr/etc.php}
{Electronics and Communication Engineering}
\hfill
\DatestampY{2011} --
\DatestampY{2015}
\begin{Detail}
\SubBulletItem
Thesis: 
Target Tracking using Parametric Spectral Estimation Methods
\SubBulletItem
GPA: 9.02 / 10.00
\end{Detail}

%%%%%%%%%%%%%%%%%%%%%%%%%
%% RESEARCH EXPERIENCE %%
%%%%%%%%%%%%%%%%%%%%%%%%%

\Section
{Research Experience}
{Experience}
{PDF:ResearchExperience}

\Entry
{\textbf{Graduate Student Researcher}}
\hfill
\DatestampYMD{2018}{09}{15} --
Present

\Gap
\BulletItem
University of California, Riverside
\hfill
CA, USA
\begin{Detail}
\SubBulletItem
Group:
Video Computing Group 
\SubBulletItem
Supervisors:
Dr.~Amit K. Roy-Chowdhury
\SubBulletItem
Focus:
Computer Vision and Deep Learning.
\end{Detail}

\Entry
{\textbf{Research Scholar}}
\hfill
\DatestampYMD{2016}{02}{15} --
\DatestampYMD{2018}{04}{15}

\Gap
\BulletItem
National Institute of Technology, Tiruchirappalli
\hfill
Tamil Nadu, India
\begin{Detail}
\SubBulletItem
Group:
Signal and Image Processing Lab.
\SubBulletItem
Supervisor:
Dr.~P.~Palanisamy 
\SubBulletItem
Focus:
Array Signal Processing, Compressed Sensing.
\end{Detail}

\Entry
{\textbf{Research Assistant}}
\hfill
\DatestampYMD{2014}{05}{15} --
\DatestampYMD{2015}{08}{15}

\Gap
\BulletItem
Silicon Institute of Technology, Bhubaneswar
\hfill
Odisha, India
\begin{Detail}
\SubBulletItem
Supervisor:
Prof.~Utpal~K.~Dash
\SubBulletItem
Focus:
Array Signal Processing.  
\end{Detail}

%%%%%%%%%%%%%%%%%%%%%%%%%
%% TEACHING EXPERIENCE %%
%%%%%%%%%%%%%%%%%%%%%%%%%

\Section
{Teaching Experience}
{Experience}
{PDF:ResearchExperience}
\Entry
{\textbf{Teaching Assistant}}
\hfill
\DatestampYMD{2019}{09}{15} --
\DatestampYMD{2020}{03}{15}

\Gap
\BulletItem
University of California, Riverside
\hfill
CA, USA
\begin{Detail}
\SubBulletItem
Under-Graduate Course: 
Senior Design Project (Computer Vision) (EE175A/EE175B)
\SubBulletItem
Supervisor:
Dr.~Amit K. Roy-Chowdhury   
\end{Detail}

\Entry
{\textbf{Teaching Assistant}}
\hfill
\DatestampYMD{2018}{01}{15} --
\DatestampYMD{2018}{04}{15}

\Gap
\BulletItem
National Institute of Technology, Tiruchirappalli
\hfill
Tamil Nadu, India
\begin{Detail}
\SubBulletItem
Graduate Course: 
Digital Signal and Image Processing Lab. (EC610)
\SubBulletItem
Supervisor:
Dr.~P.~Palanisamy   
\end{Detail}

%%%%%%%%%%%%%%%%%%
%% PUBLICATIONS %%
%%%%%%%%%%%%%%%%%%

\Section
{Selected Publications}
{Publications}
{PDF:Publications}

\begingroup
\renewcommand{\MaxNumberedItem}{[88]}
%%%%%%%%%%%%%%%%%%%%%%%%%%%%%%%%
%% Conference
%%%%%%%%%%%%%%%%%%%%%%%%%%%%%%%%
\NumberedItem{[1]}
\href{https://arxiv.org/abs/2003.09565}
{\underline{Abhishek~Aich}*, Akash Gupta*, Rameswar Panda, Rakib Hyder, Salman Asif, and Amit Roy-Chowdhury,
``Non-Adversarial Video Synthesis with Learned Priors,''
IEEE CVPR, 
\DatestampY{2020}.} (* joint first authors)

\Gap
\NumberedItem{[2]}
Akash Gupta, \underline{Abhishek~Aich}, Kevin Rodriguez, G. Venugopala Reddy, and Amit Roy-Chowdhury,
``Deep Quantized Representation for Enhanced Reconstruction,''
ISBI 2020 Workshop, 
\DatestampY{2020}. 

\Gap
\NumberedItem{[3]}
\href{https://ieeexplore.ieee.org/document/8228078}
{\underline{Abhishek~Aich}, and P.~Palanisamy,
``A novel CS beamformer root-MUSIC algorithm and its subspace deviation analysis,''
in \textit{IEEE Region 10 Conference (TENCON)},
Penang, Malaysia, pp. 1404-1408,
\DatestampY{2017}.}

\Gap
\NumberedItem{[4]}
\href{https://ieeexplore.ieee.org/document/8286749}
{\underline{Abhishek~Aich}, and P.~Palanisamy,
``On application of OMP and CoSaMP algorithms for DOA estimation problem,''
in \textit{IEEE International Conference on Communication and Signal Processing (ICCSP)},
Chennai, India, 
\DatestampY{2017}.} (\textit{Oral})

\Gap
\NumberedItem{[5]}
\href{https://ieeexplore.ieee.org/document/7848508}
{\underline{Abhishek~Aich}, and P.~Palanisamy,
``A strict bound for dimension of measurement matrix for CS beamformer MUSIC algorithm,''
in \textit{IEEE Region 10 Conference (TENCON)},
Singapore, pp. 2602-2605,
\DatestampY{2016}.} (\textit{Oral})

\endgroup

%%%%%%%%%%%%%%%%%%%%%%%%%%%
%%%%%%%% Projects %%%%%%%%%
%%%%%%%%%%%%%%%%%%%%%%%%%%%
\Section
{Projects}
{Projects}
{PDF:Projects}

\Entry
{\textbf{Video Generation from Learned Priors}}
\hfill
\DatestampYMD{2019}{07}{15} --
\DatestampYMD{2019}{11}{15}
\begin{Detail}
\SubBulletItem
Supervisor:
Dr.~Amit K. Roy-Chowdhury   
\end{Detail}
\Gap
\BulletItem \textbf{Goal}: Generate short video clips without pixel inputs.
\BulletItem Designed a generative network to generate the realistic videos using learnable latent vectors, using non-adversarial approach.
\BulletItem Introduced a novel triplet condition on the latent vectors to get good latent vector representation of video frames.

\BigGap
\Entry
{\textbf{Multi-View video frame prediction using STAR-GAN}}
\hfill
\DatestampYMD{2019}{03}{15} --
\DatestampYMD{2019}{06}{15}
\begin{Detail}
\SubBulletItem
Supervisor:
Dr.~Amit K. Roy-Chowdhury   
\end{Detail}
\Gap
\BulletItem \textbf{Goal}: Predict missing frames in one camera view using other reference camera views. 
\BulletItem Designed a STAR-GAN based model to predict missing frames in one camera by using view-parallel frames from other reference cameras. 
\BulletItem Introduced a novel cycle consistency based loss for learning a weighted relationship between missing frame and corresponding reference frames from other cameras. 

\BigGap
\Entry
{\textbf{Continual Learning in Person Re-ID systems}}
\hfill
\DatestampYMD{2019}{01}{15} --
\DatestampYMD{2019}{03}{15}
\begin{Detail}
\SubBulletItem
Supervisor:
Dr.~Amit K. Roy-Chowdhury   
\end{Detail}
\Gap
\BulletItem \textbf{Goal}: Design a global Person-ReID system to work for different places without forgetting previous data distribution   
\BulletItem Designed a deep generative network based model to allow a Person Re-ID system to continuously learn different scenarios (in this case, different datasets) without forgetting past person identities in different conditions.

%%%%%%%%%%%%%%%%%%%%%%%%%%%
%% AWARDS & SCHOLARSHIPS %%
%%%%%%%%%%%%%%%%%%%%%%%%%%%

\Section
{Awards \&\newline
Scholarships}
{Awards \& Scholarships}
{PDF:AwardsAndScholarships}

\BulletItem
\textbf{Deans Distinguished Fellowship Award}, University of California, Riverside
\hfill
\DatestampY{2018} --
\DatestampY{2019}

\Gap
\BulletItem
\textbf{MHRD Scholarship}, Govt. of India
\hfill
\DatestampY{2016} --
\DatestampY{2018}

\Gap
\BulletItem
\textbf{Scholar's Club}, Silicon Institute of Technology, Bhubaneswar
\hfill
\DatestampY{2012} --
\DatestampY{2015}
\begin{Detail}
\SubBulletItem For being in the Top 3 of the Electrical and Communication Engineering Department
\end{Detail}

\Gap
\BulletItem
\textbf{e-Medhabruti Scholarship}, Govt. of Odisha
\hfill
\DatestampY{2012} --
\DatestampY{2015}

%%%%%%%%%%%%
%% SKILLS %%
%%%%%%%%%%%%

\Section
{Technical Skills}
{Technical Skills}
{PDF:TechnicalSkills}

\Entry
\BulletItem \textbf{Programming Skills}: Python, MATLAB
\BulletItem \textbf{Deep Learning Libraries}: PyTorch \BulletItem \textbf{Scientific Computing Libraries}: numpy, scipy, sciKit-learn, matplotlib
\BulletItem \textbf{Others}: \LaTeX, MS Office, OpenCV, Jupyter

%%%%%%%%%%%%%%%%%%%%%%
%% Graduate Courses %%
%%%%%%%%%%%%%%%%%%%%%%

\Section
{Graduate Courses}
{Graduate Courses}
{PDF:GraduateCourses}

\Entry
 $\bullet$ Introduction to Deep Learning $\bullet$ Adv. Computer Vision $\bullet$ Machine Learning $\bullet$ Information Theory $\bullet$ Convex Optimization $\bullet$ State and Parameter Estimation Theory $\bullet$ Stochastic Processes $\bullet$ Sparsity, Structure, and Inference $\bullet$ Math. Methods for EE $\bullet$ Adv. Digital Signal Processing


%%%%%%%%%%%%%%%%%%%%%%%%%%%%%%%%%%%%%%%%%%%%
%% PROFESSIONAL AFFILIATIONS & ACTIVITIES %%
%%%%%%%%%%%%%%%%%%%%%%%%%%%%%%%%%%%%%%%%%%%%

\Section
{Professional Activities}
{Professional Activities}
{PDF:ProfessionalActivities}

\Entry
\textbf{Conference Reviewer:}
\newline
IEEE TENCON 2016, IEEE TENCON 2017

\Entry
\textbf{Journal Reviewer:}
\newline
IEEE Transactions on Signal Processing, Taylor \& Francis
International Journal of Electronics Letters, IET Signal Processing


%%%%%%%%%%%%%%%%%%%%%%%%%%%%%%%%%%%%%%%%
%% THIS IS A SECTION WITH USAGE NOTES %%
%%%%%%%%%%%%%%%%%%%%%%%%%%%%%%%%%%%%%%%%
\begin{comment}
% Declare a new group to limit the scope of \color to this section.
\begingroup
\color{blue}

\Section
{This is a\newline
Section\newline
With\newline
Usage Notes}
{This is a Section With Usage Notes (For PDF Bookmark)}
{PDF:ThisIsASectionWithUsageNotes:ForPDFLink}

\SubSection
{This is a SubSection}
{This is a SubSection (For PDF Bookmark)}
{PDF:ThisIsASubSection:ForPDFLink}

\BigGap
\BulletItem
Use \CodeCommand{Section\{a\}\{b\}\{c\}} and
\CodeCommand{SubSection\{a\}\{b\}\{c\}}
to create sections and subsections, where
\Code{a} is the heading displayed on the page,
\Code{b} is the PDF bookmark heading, and
\Code{c} is the internal PDF link (must be unique).
Sections and subsections will appear in the PDF bookmarks.
Note the CamelCase command names.

\Gap
\BulletItem
Use
\CodeCommand{Entry},
\CodeCommand{BulletItem},
\CodeCommand{SubBulletItem},
\CodeCommand{SubSubBulletItem},
\CodeCommand{Item},
\CodeCommand{SubItem},
\CodeCommand{SubSubItem},
\CodeCommand{NumberedItem},
etc.,
to create entries in the main body of the CV.

\Gap
\BulletItem
Enclose entry details between
\CodeCommand{begin\{Detail\}} and
\CodeCommand{end\{Detail\}}
so that they are typeset in a smaller font.
\begin{Detail}
\Item
This is an example of entry detail text enclosed in a \Code{Detail} environment.
\end{Detail}

\Gap
\BulletItem
Use \CodeCommand{Gap} and \CodeCommand{BigGap} to insert vertical spaces between entries to improve layout.

\BigGap
\SubSection
{This is Another SubSection}
{This is Another Subsection (For PDF Bookmark)}
{PDF:ThisIsAnotherSubSection:ForPDFLink}

\BigGap
\Entry
This is a plain \CodeCommand{Entry},
followed by an \CodeCommand{hfill} and a date range
\hfill
\DatestampYM{2015}{10} --
\DatestampYM{2015}{12}

\Gap
\BulletItem
This is a \CodeCommand{BulletItem}.
\Item
This is an \CodeCommand{Item}, which has no bullet.
Note the alignment with the \CodeCommand{BulletItem} above.

\Gap
\SubBulletItem
This is a \CodeCommand{SubBulletItem}.
\SubItem
This is a \CodeCommand{SubItem}, which has no bullet.
Note the alignment with the \CodeCommand{SubBulletItem} above.

\Gap
\SubSubBulletItem
This is a \CodeCommand{SubSubBulletItem}.
\SubSubItem
This is a \CodeCommand{SubSubItem}, which has no bullet.
Note the alignment with the \CodeCommand{SubSubBulletItem} above.

\Gap
\NumberedItem{[42]}
This is a \CodeCommand{NumberedItem}.
Change the value of the macro \CodeCommand{MaxNumberedItem} to adjust the indentation width.

\BigGap
\SubSection
{Line, Paragraph, and Page Breaks}
{Line, Paragraph, and Page Breaks (For PDF Bookmark)}
{PDF:LineParagraphAndPageBreaks:ForPDFLink}

\BigGap
\BulletItem
To create a new line within the same paragraph (i.e., preserving the same paragraph indentation), use \CodeCommand{newline} instead of \CodeCommand{\textbackslash};
the latter will reset the paragraph indentation.

\Gap
\BulletItem
To create a new paragraph, use \CodeCommand{par} or simply leave an empty line.
Paragraph indentations (from
\CodeCommand{Entry},
\CodeCommand{BulletItem},
\CodeCommand{SubBulletItem},
\CodeCommand{SubSubBulletItem},
\CodeCommand{Item},
\CodeCommand{SubItem},
\CodeCommand{SubSubItem},
\CodeCommand{NumberedItem},
etc.) do not carry across different paragraphs.

\Gap
\BulletItem
To create a new page, use \CodeCommand{newpage}.

\BigGap
\SubSection
{Dates}
{Dates (For PDF Bookmark)}
{PDF:Dates:ForPDFLink}

\BigGap
\BulletItem
Use the following macros to specify and display dates consistently:
\SubBulletItem
\CodeCommand{DatestampYMD\{yyyy\}\{MM\}\{dd\}}
(e.g., \CodeCommand{DatestampYMD\{2008\}\{01\}\{15\}})
\SubBulletItem
\CodeCommand{DatestampYM\{yyyy\}\{MM\}}
(e.g., \CodeCommand{DatestampYM\{2008\}\{01\}})
\SubBulletItem
\CodeCommand{DatestampY\{yyyy\}}
(e.g., \CodeCommand{DatestampY\{2008\}})

\Gap
\BulletItem
Change the date format option passed to the document class to adjust how dates are displayed throughout the document:
\SubBulletItem
\Code{MMMyyyy} (``Jan~2008'')
\SubBulletItem
\Code{ddMMMyyyy} (``15~Jan~2008'')
\SubBulletItem
\Code{MMMMyyyy} (``January~2008'')
\SubBulletItem
\Code{ddMMMMyyyy} (``15~January~2008'')
\SubBulletItem
\Code{yyyyMMdd} (``2008-01-15'')
\SubBulletItem
\Code{yyyyMM} (``2008-01'')
\SubBulletItem
\Code{yyyy} (``2008'')

\endgroup
\end{comment}

\end{Body}


\end{document}
